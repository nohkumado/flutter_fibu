%*****************************************************************************
%\originalTeX
%\input{psfig}                     %solange der postscript includer da ist
%\germanTeX
%%%%%%%%%%%%%%%%

\def\figurepath{./}
\def\psfigurepath#1{\edef\figurepath{#1}}
%\def\docpath{scsi::4.$.word.texfiles.diplarbeit.bilder.}
%\def\docpath{/home/bboett/tex/gb.dipl/bilder/}

%%%%%% aus psfig

%\psfigurepath{scsi::4.$.word.texfiles.diplarbeit.bilder.}
%\psfigurepath{/home/bboett/tex/gb.dipl/bilder/}
%*****************************************************************************
% pageheight = 29.7cm (= DIN A4)

\topmargin-0.7cm                    % oberer Rand=3.7 +tm=2cm
\headheight1cm                     % Hoehe  der Kopfzeile 1cm
\headsep0.3cm                       % Abstand zur Text/Kopfzeile
\topskip1cm                         % Abstand Textanf/Text
%\footheight0.5cm                    % Hoehe Fu"sleiste .5cm
\textheight23cm                     % L"ange Seite
\footskip2cm                        % Tiefe Fu"szeile
% pagewidth =  21.0cm (= DIN A4)
\evensidemargin1cm                  % linker Rand = 3 +1 =4cm
\oddsidemargin1cm                   %    "                 "
\textwidth15cm                      % Textbreite





%*****************************************************************************
% Centred scaled captioned labeled figure
% example: \psfig{file.ps}{scale}{caption}{reference}
% \def\psfig#1#2#3#4{
%        \begin{figure}
%                \def\epsfsize##1##2{#2##1}
%                \centerline{\epsfbox{#1}}
%                \caption{#3}
%                \label{#4}
%        \end{figure}
% }

\usepackage{myepsf}
\def\psfig#1#2#3#4{
       \begin{figure}
               \setepsfsize{#2}\centerline{\epsfbox{#1}}
               \caption{#3} \label{#4}
       \end{figure}
}

%
%\newcommand{\postfig}[4]{       %for use with psfig
%   \begin{figure}[hbtp]
%   \centerline{\hbox{\psfig{figure=#1, height=#2}}}
%   \caption{#3}\label{#4}
%   \end{figure}}



%\input{\docpath hyph_ge}
%selb-st"an-dig
                                                                            
%*****************************************************************************
%\newtheorem{definition}{Definition}[chapter]
%*************** Damit die Bilder da bleiben wo ich sie hingepackt habe ******
\renewcommand{\topfraction}{0,9999}
\renewcommand{\bottomfraction}{0,9999}
\renewcommand{\textfraction}{0}

%*****************************************************************************

