%*****************************************************************************
% copyright Winfried and Bruno Böttcher
\documentclass[12pt]{report}
\usepackage[italian]{babel}
\usepackage{isolatin1}
\usepackage{epsf}
\usepackage{html}

\pagestyle{headings}
%*****************************************************************************
%*****************************************************************************
%\originalTeX
%\input{psfig}                     %solange der postscript includer da ist
%\germanTeX
%%%%%%%%%%%%%%%%

\def\figurepath{./}
\def\psfigurepath#1{\edef\figurepath{#1}}
%\def\docpath{scsi::4.$.word.texfiles.diplarbeit.bilder.}
%\def\docpath{/home/bboett/tex/gb.dipl/bilder/}

%%%%%% aus psfig

%\psfigurepath{scsi::4.$.word.texfiles.diplarbeit.bilder.}
%\psfigurepath{/home/bboett/tex/gb.dipl/bilder/}
%*****************************************************************************
% pageheight = 29.7cm (= DIN A4)

\topmargin-0.7cm                    % oberer Rand=3.7 +tm=2cm
\headheight1cm                     % Hoehe  der Kopfzeile 1cm
\headsep0.3cm                       % Abstand zur Text/Kopfzeile
\topskip1cm                         % Abstand Textanf/Text
%\footheight0.5cm                    % Hoehe Fu"sleiste .5cm
\textheight23cm                     % L"ange Seite
\footskip2cm                        % Tiefe Fu"szeile
% pagewidth =  21.0cm (= DIN A4)
\evensidemargin1cm                  % linker Rand = 3 +1 =4cm
\oddsidemargin1cm                   %    "                 "
\textwidth15cm                      % Textbreite





%*****************************************************************************
% Centred scaled captioned labeled figure
% example: \psfig{file.ps}{scale}{caption}{reference}
% \def\psfig#1#2#3#4{
%        \begin{figure}
%                \def\epsfsize##1##2{#2##1}
%                \centerline{\epsfbox{#1}}
%                \caption{#3}
%                \label{#4}
%        \end{figure}
% }

\usepackage{myepsf}
\def\psfig#1#2#3#4{
       \begin{figure}
               \setepsfsize{#2}\centerline{\epsfbox{#1}}
               \caption{#3} \label{#4}
       \end{figure}
}

%
%\newcommand{\postfig}[4]{       %for use with psfig
%   \begin{figure}[hbtp]
%   \centerline{\hbox{\psfig{figure=#1, height=#2}}}
%   \caption{#3}\label{#4}
%   \end{figure}}



%\input{\docpath hyph_ge}
%selb-st"an-dig
                                                                            
%*****************************************************************************
%\newtheorem{definition}{Definition}[chapter]
%*************** Damit die Bilder da bleiben wo ich sie hingepackt habe ******
\renewcommand{\topfraction}{0,9999}
\renewcommand{\bottomfraction}{0,9999}
\renewcommand{\textfraction}{0}

%*****************************************************************************


%*****************************************************************************
\begin{document}

\tableofcontents

\chapter{Introduzzione}

\section{Perche fare una contabilita' ?}
La gestione  di trasazioni legati a movimenti di valori rende  necessaria una
contabilita' scritta. WB-FIBU e' un programma per ordinatore elettronico, che
applica il cosidetto sistema della partita doppia.  

Questo sistema fu' descritto  per la  prima volta nel 1445 da Luca Paciolo e 
si basa su una idea molto semplice:
"Ogni movimento o trasazione viene considerato come un movimento finanziario,
che nasce in una sorgente, e che sparisce in un pozzo."
Le sorgenti e i pozzi (chiamati conti) possono essere immaginati come 
recipienti disposti su un grande tavolo, dai  quali si prende un importo qua 
(sorgente), per  metterlo altrove (pozzo) -  ma sempre  rimanendo sul grande 
tavolo.
Da un punto di visto aritmetico, l'estrazione di soldi da una sorgente e' una
sostrazione (-), mentre la deposizione in un pozzo e' una addizione (+).

\begin{verbatim}
      Pozzo                     Sorgente
      +      <--- Causale ---   -
\end{verbatim}

Siccome ogni operazione concerne sempre 2 conti, questo sistema si chiama
partita doppia.

Il Piano dei Conti e' l'insieme dei conti liberamente scelti per descrivere i
sotto-insiemi da distinguere. Questo Piano puo' essere ampliato in ogni mo- 
mento dell'esercizio. La cancellazione di un conto invece e' solamente pos- 
sibile, se non esiste alcun movimento su questo conto.

L'insieme dei conti e' suddiviso in 4 gruppi:

\begin{itemize}
   \item Conti Attivita' (Cassa, conti correnti, Immobili, Crediti a terzi, etc)
\item Conti Passivita'(patrimonio,riserve, debiti verso terzi, etc)
\item Conti Spese   (diversificazione a gusto personale) 
\item Conti Entrate (diversificazione a gusto personale) 
\end{itemize}

E' necessario un po di raggionamento astatto per capire o per accettare che
i saldi di questi conti hanno segno aritmetico alternato:

\begin{itemize}
   \item i conti Attivita'  hanno segno positivo
\item i conti Passivita' hanno segno negativo
\item i conti Spese      hanno segno positivo
\item i conti Entrate    hanno segno negativo
\end{itemize}

Cosi la somma globale di tutti i conti e' sempre uguale a zero.

Il Giornale e' la registrazione sequenziale di tutti i movimenti. Il compito
importante e principale del contabile e' di interpretare la transazione e di
assegnare i 2 conti interessati. In caso di dubbio, si puo' anche creare un
conto ex nuovo per trovare una sistemazione adeguata per il movimento.
 
\subsubsection{L'estratto conto:} per tutti i movimenti dell'esercizio
corrente, che si trovano nel Giornale, si puo' fare un ragruppamento dei soli
movimenti che interessano un determinato conto, sia in positivo, che in
negativo. Questo sotto-insieme del Giornale si chiama estratto conto. La somma
aritmetica del saldo iniziale, e di tutti gli importi in piu' o in meno,
produce un saldo finale. Il significato di un estratto particolare dipende
della diversifica- zione del titolo di conto. Un conto spese nel quale
finiscono spese di energia elettrica e di gas, ha meno significato che un conto
che evidenzia solo energia elettrica, p.e. 

\subsubsection{Bilancio / Conto Economico:} sono due cose completamente
diversi.  Il Bilancio e' la comparazione fra conti Attivita' e conti Passivita'
in un determinato giorno dell'esercizio. Esso da un idea sulla situzione
patrimoniale.

Il conto economico invece e' la comparazione fra conti Spese e conti Entrate,
per un periodo determinato, normalmente dall'inizio esercizio fino al giorno 
di oggi. Esso da un idea sulla situazione economica dell'esercizio in corso.

\subsubsection{Valori iniziali:} all'inizio dell'esercizio bisogna dare valori
iniziali ai soli conti Attivita' e Passivita'in modo tale, che la somma dei
conti positivi attivi e' uguale alla somma negativa dei conti passivi.  (+x -x
= 0 complessivamente).

Siccome i movimenti che si aggiungono durante l'esercizio interessano sempre
due conti, una volta in +, una volta in -, l'equilibrio iniziale si mantiene,
e la somma di tutti i conti rimane sempre uguale a zero, se tutto e' stato 
registrato correttamente.

\section{Perche' WB-FIBU ?}

Leggendo i tanti libri in materia, si scopre ben presto, che le regole di 
contabilita', parzialmente anche sotto forma di disposizioni di legge, sono
piu' complicati del necessario. E questo si ripercuote nei tanti programmi
per PC o altre piattaforme ottenibile anche a basso prezzo.

Il WB-FIBU invece cerca di realizzare semplicemente la partita doppia, senza 
restrizioni qualsiase p.e. permette di togliere o rettificare linee giornale, 
che non sono corretti, oppure accetta registrazioni con data anteriore per
fatti annunciati p.e. dalla banca - il giornale viene sorteggiato autom. in 
ordine di data crescente. Cosi si ottiene un giornale sempre pulito e' cor- 
retto e di conseguenza piu' chiaro.

Il programma doveva essere concepito in modo da eseguirsi su ogni PC 
standard, in modo portabile su altre piattaforme, e senza requisiti onerosi 
in termini di programmi aggiuntivi oltre al sistema operativo della machina.

Il gruppo di utenza visata: privati, contabili di club o associazioni piccoli 
senza obblighi di legge.

Il modo di lavorare con WB-FIBU:
\begin{itemize}
\item si fa partire il programma sotto DOS, oppure sotto Windows;

\item si entrano i dati nuovi del giornale a partire  della tastiera,
usando  l'editore incorporato, che permette di copiare  campi  invariati  dalla
linea precedente, e che produce il formato giusto dei dati;  

\item Mentre  in  una  contabilita'  manuale,  questa scrittura  deve
essere ripetuto due volte su ogni scheda conto, con i calcoli dei saldi, qui si
sceglie l'opzione XEQ,  e in pochi secondi il  programma  produce quello che e'
richiesto: stampa Piano dei Conti,  stampa Giornale,  tutti  gli estratti,
Bilancio e Conto Economico.

\item I risultati possono essere visualizzati allo schermo con VIEW - le
linee degli estratti sono allora ridotti per stare in una linea video. Ogni
tanto e' utile di fare una stampa scegliendo PRINT. 

\item In chiusura il programma richiede un dischetto per fare una copia di
backup per sicurezza, solo se i dati furono cambiati. Si consiglia di usare
almeno 2 dischetti in alternanza, per ridurre il rischio del dischetto
rovinato, e per disporre sempre delle due ultime situazioni.  Cosi in caso di
rottura del discho rigido della machina, si puo' conti- nuare subito su ogni
altra machina a disposizione.

\item L'insieme dati con i risultati rimane sempre valido fino ad una nuova
esecuzione secondo il punto 3 di sopra. E' allora possibile, di interro- gare
la contabilita' esattamente come fogliare uno stampato.  Questo dovrebbe
permettere di ridurre l'uso della carta ad un minimo.

\end{itemize}

\subsection{Utilizzatori potentiali di bb-Fibu:}
Proprieta' WB-FIBU:
\begin{itemize}

\item Particolari, contabili di piccole associazioni o piccole imprese senza costrette legali.

\item Programma scritto in C (Turbo C/PC) portabile su diverse piattaforme:
PC-DOS, Archimedes, UNIX WS, ..;

\item Il linguaggio di dialogo nonche' quello usato per la rappresentazione
dei risultati e' definito in un insieme esterno al programma, e
conseguentemente a scelta, se esiste la traduzione. Attualmente esistono:
fibu.de, fibu.it, ...

\item Il Piano dei Conti e' liberamente definibile;

\item La tabella dei cambi di valuta e' prevista (non ancora implementato);

\item L'entrata dei dati sotto controllo di un editore linea adattato;

\item I tasti di funzione del editore sono definiti in un file esterno
(fibu.cfg);

\item Tutti i dati sono elaborati in memoria viva;

\item Il numero massimo di registrazioni dipende solamente dalla memoria a
disponibile.

\item Consulatazione dei risultati allo schermo - rimangono validi fino alla
prossima esecuzione e possono essere visualizzati a domanda;

\item Stampa su commando su LPT1, sia in locale che in rete, con adattamento
della stampante a traverso commani ESC in HP PCL contenuti in fibu.prn.  E
necessario il supporto di scrittura normale in ISO A4 landscape, oppure 16 cpi
per la stampa in ISO A4 portrait.

\end{itemize}
     
Pre-requisiti per questo programma:
\begin{itemize}
\item PC con MS-DOS, 640 Kbytes a 8bit/byte
\item a secondo della configurazione del PC sono possibile ca. 8000 righe;
\item stampante non indispensabile ma utile (16cpi,o carta a 120 colonne);
\end{itemize}

WB-FIBU insiemi dati (files):
\begin{itemize}
\item FIBU.EXE   programma per DOS
\item FIBU.DE    tutti i messaggi in tedesco
\item FIBU.IT    tutti i messaggi in italiano
\item FIBU.CFG   definizione dei tasti per l'editore
\item FIBU.PRN   addattamento stampante in HP PCL notazione
     N.B. questi insiemi devono trovarsi nella stessa directory

\item <Name>.KPL  Piano dei conti <Name>, p.e. ELKI93.KPL
\item <Name>.JRL  Giornale        <Name>, p.e. ELKI93.JRL
\item <Name>.LST  Risultati        <Name>,p.e. ELKI93.LST
     N.B. questi insiemi devono trovarsi nella stessa directory
\end{itemize}
\begin{appendix}
\chapter{Informazzioni techniche}

\section{ Formato dei dati nella X.KPL}


\begin{verbatim}
Pos 01 controllo pagina per l'estratto (P nuova pagina, spazio= stampa cont.)
    02 spazio
    03 numero conto, 1-4 posizioni, p.e. 1101 
    07 spazio
    08 Titolo del conto,           50 caratteri,  p.e. Cassa
    58 Valuta, 2 caratteri,        p.e. LI, DM, FS, etc (derzeit nicht verw.)
    60 importo preventivo  (solo per conti spesa,entrata), 12 posizioni
    72 spazio, fine linea
\end{verbatim}

\section{ Formato dei dati nella X.JRL}

\begin{verbatim}
Pos 01 Data 8 caratteri, p.e. 01.01.1993
    09 spazio 
    10 Numero conto +, (pozzo),    1-4 posizioni alineati a destra
    14 spazio
    15 Numero conto -, (sorgente), 1-4 posizioni alineati a destra
    19 spazio
    20 descrizione causale, 45 caratteri al massimo
    65 valuta 2 posizioni come da x.kpl
    67 importo 12 posizioni alineati a destra 
    79 spazio e fine linea; durante l'esecuzione, si aggianca qui il saldo
    90 l'ultima posizione del saldo (solo in memoria)
\end{verbatim}

Nei due insiemi la fine logica dei dati e' indicati con un /*EOF 

\chapter{Concetti contabili}

Si copia il contenuto del dischetto base in una directroy del disco rigido:

\section{Valori iniziali}
All'inizio dell'esercizio e' necessario dare valori iniziali ai conti 
Attivita' e Passivita', i cosidetti Riporti.
Per questi riporti e' consentito in modo eccezzionale di avere registrazioni
con un solo numero conto valido, mentre l'altro conto e' messo a zero.
Si usa il numero conto + per riportare un saldo positivo, e il numero conto -
per riportare un saldo negativo.

\paragraph{Esempio:}

\begin{verbatim}
01.01.93 1101    0 Riporto cassa            100000
01.01.93    0 2100 Riporto patrimonio       100000
\end{verbatim}

\subsection{Esempi pratici di contabilita'}

Per gli esempi serve questo piano dei conti :

\begin{verbatim}
  1001 Cassa
  1010 Conto Corrente
  1020 Conto a termine: BOT
  1100 Credito a socio x
  1999 Giroconto
     0 fine attivita'
  2001 Patrimonio
  2010 Fondo di riserva
  2100 Debito verso fornitore
     0 fine passivita'
  3001 Spese di amministrazione
  3010 Spese per l'energia
  3020 Spese per servizi
     0 fine costi
  4100 Entrata: da soci
  4500 Entrata: interessi
  4600 Entrata: diversi
     0 fine entrate
\end{verbatim}

E frequente di dover trattare dei movimenti, che in realta si compongono di
diversi movimenti parziali, che vanno essere registrati in modo detagliato.
1. incassi vari (cxxx = numero consecutivo del giustificativo, classificatore)
\begin{verbatim}
  01.02.95 1001 4100 ricevo soldi,cxxx             100000
  15.02.95 1001 4100 ricevo assegno, cxxx          500000
  01.03.95 1010 1001 versamento da cassa           600000
\end{verbatim}

2. fattura entrante, pagamento dilazionato (dxxx = numero pezzo giustificativo)
\begin{verbatim}
  01.02.95 3010 2100 fatt ENEL 12-1.95,dxxx        300000
  15.03.95 2100 1010 pagamento bolletta ENEL       300000
\end{verbatim}

3. fattura entrante, pagamento immediato, ripartizione spesa
\begin{verbatim}
  01.02.95 1999 1010 fatt Telecom 1bim95, dxxx     600000
  01.02.95 3020 1999 fatt Telecom 1bim95 parte uff 450000
  01.02.95 1100 1999 fatt Telecom 1bim95 parte pri 150000
  15.02.95 1001 1100 da x / telefono               150000
\end{verbatim}

4. acquisto BOT con interessi scontati
\begin{verbatim}
  01.02.95 1020 1010 acquisto BOT 010595 1M  920000

  01.05.95 1010 1999 rimborso BOT 010595 1M 1000000
  01.05.95 1999 1020 rimborso BOT 010595 1M  920000
  01.05.95 1999 4500 interess BOT 010595 1M   80000
\end{verbatim}

Questa tecnica permette di dettagliare come si vuole, e di nascondere questo
su quei estratti conti, dove c'e bisogno di mantenere esattamente i stessi
movimenti cha fara la banca e come appaiano nel estratto conto della banca.

Finalmente si racommanda di usare sempre le stesse espressioni, corti, 
significativi, per le stesse cose. Sempre guardare nel giornale piu' in alto
per descrivere lo stesso fatto in modo uguale.


Winfried, Bruno Böttcher
November 1995
\end{appendix}
\end{document}

