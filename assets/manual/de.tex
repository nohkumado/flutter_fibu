%*****************************************************************************
% copyright Winfried and Bruno Böttcher
\documentclass[12pt]{report}
\usepackage[german]{babel}
%\usepackage{isolatin1}
\usepackage{epsf}
\usepackage{html}

\pagestyle{headings}
%*****************************************************************************
%*****************************************************************************
%\originalTeX
%\input{psfig}                     %solange der postscript includer da ist
%\germanTeX
%%%%%%%%%%%%%%%%

\def\figurepath{./}
\def\psfigurepath#1{\edef\figurepath{#1}}
%\def\docpath{scsi::4.$.word.texfiles.diplarbeit.bilder.}
%\def\docpath{/home/bboett/tex/gb.dipl/bilder/}

%%%%%% aus psfig

%\psfigurepath{scsi::4.$.word.texfiles.diplarbeit.bilder.}
%\psfigurepath{/home/bboett/tex/gb.dipl/bilder/}
%*****************************************************************************
% pageheight = 29.7cm (= DIN A4)

\topmargin-0.7cm                    % oberer Rand=3.7 +tm=2cm
\headheight1cm                     % Hoehe  der Kopfzeile 1cm
\headsep0.3cm                       % Abstand zur Text/Kopfzeile
\topskip1cm                         % Abstand Textanf/Text
%\footheight0.5cm                    % Hoehe Fu"sleiste .5cm
\textheight23cm                     % L"ange Seite
\footskip2cm                        % Tiefe Fu"szeile
% pagewidth =  21.0cm (= DIN A4)
\evensidemargin1cm                  % linker Rand = 3 +1 =4cm
\oddsidemargin1cm                   %    "                 "
\textwidth15cm                      % Textbreite





%*****************************************************************************
% Centred scaled captioned labeled figure
% example: \psfig{file.ps}{scale}{caption}{reference}
% \def\psfig#1#2#3#4{
%        \begin{figure}
%                \def\epsfsize##1##2{#2##1}
%                \centerline{\epsfbox{#1}}
%                \caption{#3}
%                \label{#4}
%        \end{figure}
% }

\usepackage{myepsf}
\def\psfig#1#2#3#4{
       \begin{figure}
               \setepsfsize{#2}\centerline{\epsfbox{#1}}
               \caption{#3} \label{#4}
       \end{figure}
}

%
%\newcommand{\postfig}[4]{       %for use with psfig
%   \begin{figure}[hbtp]
%   \centerline{\hbox{\psfig{figure=#1, height=#2}}}
%   \caption{#3}\label{#4}
%   \end{figure}}



%\input{\docpath hyph_ge}
%selb-st"an-dig
                                                                            
%*****************************************************************************
%\newtheorem{definition}{Definition}[chapter]
%*************** Damit die Bilder da bleiben wo ich sie hingepackt habe ******
\renewcommand{\topfraction}{0,9999}
\renewcommand{\bottomfraction}{0,9999}
\renewcommand{\textfraction}{0}

%*****************************************************************************

                                                                
%*****************************************************************************
\begin{document}

\tableofcontents

\chapter{Einführung}

\section{Warum Finanzbuchhaltung ?}

Für die Verwaltung wertbezogener Vorgänge ist eine Finanz-Buchhaltung
unerläßlich. WB-FIBU ist ein EDV-Programm, das nach dem Prinzip der doppelten
Kontoführung arbeitet.

Die doppelte Kontoführung wurde erstmals 1445 von Luca Paciolo beschrieben und
beruht auf einer denkbar einfachen Idee: ''Jeder Vorgang wird als eine
Geldbewegung dargestellt, die von einer Quelle ausgeht und in einer Senke
verschwindet.'' Quellen und Senken (Konten genannt) kann man sich auch als
Töpfe vorstellen, die alle auf einem großen Tisch stehen, und aus denen Geld
entnommen wird (Quellen), oder in die Geld hineingelegt wird (Senken).
Arithmetisch gesehen bedeutet die Geldentnahme eine Subtraktion (-), und ein
Hineinlegen eine Addition (+).  

\begin{verbatim}
      Senke                     Quelle
      +      <--- Vorgang ---   -
\end{verbatim}

Da jeder Vorgang immer 2 Konten betrifft, heißt das Verfahren auch doppelte
Kontoführung (partita doppia).

Der Kontoplan (Konto-Rahmen) enthält eine frei wählbare Menge an Konten,wie
sie zur Beschreibung des Geschehens benötigt werden. Der Kontoplan kann
jederzeit nach Bedarf erweitert werden. Das Wegnehmen eines Kontos inmitten
eines Geschäftsjahres ist nur dann statthaft, wenn es keine Bewegungen über
dieses Konto gegeben hat.

Die Konten sind übrigens in 4 Gruppen eingeteilt:

\begin{itemize}

\item Aktiv-Konten (Kasse, Bankkonten, Immobilien-Werte, Guthaben, etc)

\item Passiv-Konten(effektives Vermögen, Patrimonio, Schulden, etc)

\item Ausgaben

\item Einnahmen

\end{itemize}

Es ist ein wenig abstraktes Denken nötig um zu verstehen oder zu glauben, daß
die Salden dieser Gruppen im Rahmen der Buchhaltung wechselnde Vorzeichen
haben:

   
\begin{itemize}

\item Aktivkonten sind positiv, 

\item Passivkonten sind negativ,

\item Ausgaben sind positiv,

\item Einnahmen sind negativ.

\end{itemize}

Dies ergibt eine globale Saldensumme von Null als Kontrolle.

Das Journal ist eine sequentielle Aufzeichnung aller Vorgänge. Die wichtige
Aufgabe des Buchhalters besteht darin, den Vorgang evt. in seine Einzelschritte
zu zerlegen und die zugehörigen Konten zu bestimmen. Im Zweifelsfall kann
jederzeit ein neues Konto eröffnet werden.

\subsection{Der Kontoauszug:} 

Alle im Journal eingetragenen Bewegungen, die ein bestimmtes Konto betreffen,
lassen sich auf einer Kontokarte zusammenfassen, und aus dem Anfangsbetrag und
den Bewegungen läßt sich der aktuelle Endbetrag des Kontos errechnen (Saldo).
Die Aussagekraft dieser Zusammenstellung hängt einzig und allein davon ab, wie
fein der Kontoplan Unterscheidungen zuläßt.

\subsection{Bilanz / Wirtschaftsrechnung:} 

Dürfen nicht verwechselt werden.  Die Bilanz ist der Vergleich der Salden von
allen Aktiv- und Passiv-Konten an einem bestimmten Tag. Sie gibt Aufschluß
über den derzeitigen Vermögensstand.

Die Wirtschaftsrechnung ist der Vergleich der Salden von allen Ausgabekonten
mit denen der Einnahmekonten, über einen bestimmten Zeitraum.  Die
Wirtschaftrechnung gibt Aufschluß über das ökonomische Verhalten.

\subsection{Anfangswerte:} 

Zu Beginn der Abrechnungsperiode (neues Journal) bekommen nur die Aktiv- und
Passivkonten Anfangswerte, und zwar so, daß die Summe der positiven
Aktivkonten gleich der Summe der negativen Passivkonten ist.  

Ein Beispiel kann im Kapitel \ref{Anfangswerte} eingesehen werden.

\subsection{Kontrollmöglichkeit:} 

Da die fortlaufend hinzukommenden Bewegungen immer 2 Konten betreffen, einmal +
(Senke), einmal - (Quelle), bleibt in jedem Zeitpunkt des Jahres die
Gesamtsumme aller Konten = Null.

Der Unterschied zwischen Aktiv- und Passivkonten ist im Betrag gleich dem
Unterschied zwischen Einnahmen und Ausgaben, jedoch mit entgegengesetztem
Vorzeichen (damit die Gesamtsumme aller Konten gleich Null wird).

\section{Warum WB-FIBU ?}

Einschlägige Bücher über Buchhaltung hinterlassen den Eindruck, daß die
begleitenden Buchhaltungsregeln, die z.T.  gesetzlich verankert sind,
  komplizierter sind, als es die Materie erfordert.

Die zahlreichen auf dem Markt befindlichen EDV-Programme tragen diesen Ballast
mit sich herum, und sind deshalb entsprechend komplex. 

Um ganz einfach die obigen Beziehungen in einem ebenso einfachen und
anspruchslosen Programm zu realisieren, wurde diese Neuentwicklung nötig (es
ist z.B. möglich, schon gemachte Journal-Einträge zu korrigieren, oder
kommende Einträge im voraus zu tätigen - das Journal wird autom. nach Datum
sortiert). Daraus resultiert eine stets bereinigte und damit klarere
Buchhaltung.

Es wurd Wert darauf gelegt, das Programm so zu schreiben, daß es leicht auf
verschiedene EDV Plattformen portiert werden kann und daß es direkt unter dem
Betriebssystem lauffähig ist.

\subsection{Zielgruppe für WB-FIBU:} 

Privatleute, Buchhalter von Vereinen oder kleinen Firmen ohne gesetzliche
Auflagen.
\chapter{Installation}
\section{Arbeitsweise mit WB-FIBU:}

\begin{itemize}

\item man startet das Program von einer Kommandolinie;

\item man gibt über die Tastatur die neuen Daten für das Journal oder den
Kontoplan ein; dabei kann man gleichbleibende Daten von der vorhergehenden
Linie kopieren (noch nicht implementiert).

\item während in einer manuellen Buchhaltung der Journal-Eintrag auf den 2
Kontokarten wiederholt werden muß (+Saldo Berechnung), wählt man einfach
Menupunkt XEQ und erzeugt eine Resultat-Datei mit dem was man braucht:
Kontoplan, Journal, alle Kontoauszüge, Bilanz und Wirtschaftsrechnung.

\item die Ergebnisse werden am Bildschirm angezeigt werden (Tab-Panel
Buchungen, Bilanz) von Zeit zu Zeit ist ein Ausdruck sinnvoll (Menupunkt PRINT,
noch nicht implementiert).

\item bei Beendigung des Programmes wird automatisch der aktuelle Stand
abgespeichert, noch sollte per Hand eine Sicherung auf Diskette gemacht werden;
man sollte 2 Disketten abwechselnd verwenden und verfügt dann immer über die
letzte und die vorhergehende Situation.  Somit kann bei Ausfall der Festplatte
oder des ganzen PC jederzeit auf einem anderen PC weitergemacht werden.

\item Die Resultat-Datei bleibt solange erhalten, bis Punkt 3 erneut
angesteuert wird. Man kann also nachträglich Kontostände abfragen, und somit
den Papierausdruck auf ein Minimum beschränken. (noch nicht implementiert)

\end{itemize}

\section{Eigenschaften WB-FIBU:}

\begin{itemize}

\item  Quell Programm in JAVA für alle EDV Plattformen, die die Java Virtual
Machine (SUN-JVM) unterstützen: PC-DOS, Archimedes, UNIX WS, ..;

\item  Dialog-Sprache und Titel für Ausgaben frei wählbar (deutsch,
ital.,franz,...) fibu*properties.

\item  Kontoplan (rahmen) frei wählbar;

\item  Fremdwährungskonten vorgesehen, Funktionen ausgeschaltet, da dieses Thema sehr komplex ist;

\item  Formular zur Eingabe der Daten, standard Operationen möglich, mit denen die Formulare vorausgefüllt werden.

\item  Abarbeitung aller Daten im Speicher;

\item  maximale Zahl der Einträge nur von der Speichergröße abhängig;

\item  Konsultation der Resultate am Bildschirm (zur Papierersparnis) - diese
bleiben bis zur nächsten Abarbeitung erhalten und können beliebig oft
angeschaut werden;


\end{itemize}

\subsection{Voraussetzungen für dieses Programm:}

\begin{itemize}
\item ein halbwegs aktueller \htmladdnormallink{apache}{http://www.apache.org/} webserver.
\item  Beliebige java-fähige Plattform, da swing benutzt wurde wird ein Hauptspeicher von mindestens 32MB empfohlen.

\item  Festplatte erwünscht - Bedarf circa 700 kbytes für das Programm;

\item  Drucker nützlich (16cpi Einstellung oder 120 Kolonnen-Papier);

\end{itemize}

\subsection{WB-FIBU Dateien:}

\begin{itemize}

\item  xF     Starter Klasse für die graphische jFibu Version.

\item  fibu\_de\_DE.properties    Meldungen und Ausgaben in deutsch

\item  fibu\_it\_IT.properties    Meldungen und Ausgaben in italienisch

\item  fibu.cfg   Editiertastenbelegungen

\item  fibu.prn   Druckeranpassung - Escape Sequenzen in HP PCL Notation N.B.
diese Dateien stehen alle im gleichen Verzeichnis

\item  <Name>.kpl   Kontoplan für Buchhaltung <Name>, z.B. ELKI93.kpl
\end{itemize}

\chapter{Benutzungshandbuch}

\section{Arbeitsweise mit bb-Fibu:}

\begin{itemize}

\item Der Zugang zum Programm erfolgt über einen Webclient, z.B. {\em lynx
    http://localhost/<PATHTOFIBU>/index.php}. Es wird ein gültiges Login verlangt, welches vom jeweiligen Buchadministrator oder vom Erstadministrator vergeben werden kann. Vom Optionsbildschirm selektiert man das Buchset von dem man arbeiten möchte.

\item Die Journaleinträge werden über das Standardoperationsformular gemacht, die vorinstallierten Standardoperationen könne einfach erweitert werden und sind an Benutzer und Buchset gebunden.

\item während in einer manuellen Buchhaltung der Journal-Eintrag auf den 2
Kontokarten wiederholt werden muß (+Saldo Berechnung), wählt man einfach
Menupunkt XEQ und erzeugt eine Resultat-Datei mit dem was man braucht:
Kontoplan, Journal, alle Kontoauszüge, Bilanz und Wirtschaftsrechnung.

\item Die Resultat-Datei bleibt solange erhalten, bis die Bilanz erneut
angesteuert wird. Man kann also nachträglich Kontostände abfragen, und somit
den Papierausdruck auf ein Minimum beschränken.

\end{itemize}

\begin{appendix}

\chapter{bb-Fibu Datei-Formate}

\subsection{Datei-Format .kpl}

\begin{verbatim}
Pos 01 Seitenkontrolle (P neue Seite, leer fortlaufend)
    02 leer
    03 Konto-Nummer 1-4 stellig, z.B. 1101 
    07 leer
    08 Konto-Titel  50 Zeichen,  z.B. Kasse
    58 Währung 2 Zeichen,        z.B. LI, DM, FS, etc (derzeit nicht verw.)
    60 Betrag für Voranschlag (0 bei Aktiv/Passivkonten), 12 Zeichen
    72 frei
\end{verbatim}

\subsubsection{Datei-Format .jrl}

\begin{verbatim}
Pos 01 Datum 8 Zeichen, z.B. 01.01.1993
    09 leer 
    10 Konto-Nummer +, (Senke),  1-4 Zeichen
    14 leer
    15 Konto-Nummer -, (Quelle), 1-4 Zeichen
    19 leer
    20 Vorgang max 45 Zeichen
    65 Währung 2 Zeichen (wie bei KPL)
    67 Betrag bis zu 12 Zeichen 
    79 frei; das Programm hängt an diese Position den Saldo an = 12 Zeichen
    90 letzte Position im Ausgabe-Format
\end{verbatim}

In fibu-Kompatibilität wird /*EOF ignoriert. 

			      \chapter{Buchhaltungskonzepte}

\subsubsection{Anfangswerte}

\label{Anfangswerte}

Am 01.01.9X müssen den Aktiv- und Passivkonten Anfangswerte gegeben werden
(sogenannte Ueberträge).  Dies geschieht dadurch, daß in diesen Linien nur
ein Konto benannt wird, entweder das positive Aktivkonto, oder das negative
Passivkonto, während das zweite Konto ausnahmsweise =0 gesetzt wird.

Beispiel:

\begin{verbatim}
01.01.93 1101    0 Uebertrag Kasse          100000
01.01.93    0 2100 Uebertrag Patrimonio     100000
\end{verbatim}

\subsubsection{Hinweise zur Buchführung}

Für die Darstellung wird folgender gekürzter Kontoplan angenommen:
\begin{verbatim}
  1001 Kasse
  1010 Kontokorrent Konto
  1020 Festgeld: BOT
  1100 Kredit an Genosse x
  1999 Zwischenkonto
     0 Ende Aktivas
  2001 Betriebsvermögen
  2010 Reservefond
  2100 Schulden an Zulieferer
     0 Ende Passiva
  3001 Verwaltungskosten
  3010 Energiekosten
  3020 Dienstleistungskosten
     0 Ende der Kosten
  4100 Einnahmen: von Mitgliedern
  4500 Einnahmen: Zinsen
  4600 Einnahmen: diverse
     0 Ende der Einnahmen
\end{verbatim}

Viele Vorgänge setzen sich in Wirklichkeit aus mehreren Einzelschritten
zusammen, die separat ins Journal eingetragen werden müssen:

\begin{itemize}

\item verschiedenen Einnahmen (cxxx = Nummer Ablage,Ordner etc)
  \begin{verbatim}
  01.02.95 1001 4100 von x,cxxx                    100000
  15.02.95 1001 4100 von y, Scheck, cxxx           500000
  01.03.95 1010 1001 von Kasse nach giro           600000
  \end{verbatim}

\item eingehende Rechnung, Zahlung später (dxxx = Nummer der Ablage,Ordner etc)
  \begin{verbatim}
  01.02.95 3010 2100 Rechnung ENEL 12-1.95,dxxx    300000
  15.03.95 2100 1010 an ENEL Ueb/Scheckxxx         300000
  \end{verbatim}

\item eingehende Rechnung, Zahlung sofort, Aufteilung der Kosten
  \begin{verbatim}
  01.02.95 1999 1010 Rech.Telecom 1bim95, dxxx     600000
  01.02.95 3020 1999 Rech.Telecom 1bim95 Dienst    450000
  01.02.95 1100 1999 Rech.Telecom 1bim95 privat    150000
  15.02.95 1001 1100 von x / priv.Telefon          150000
  \end{verbatim}

\item Festgeldanlage abgezinst
  \begin{verbatim}
  01.02.95 1020 1010 Anlage   BOT 010595 1M  920000

  01.05.95 1010 1999 Rückz.  BOT 010595 1M 1000000
  01.05.95 1999 1020 Rückz.  BOT 010595 1M  920000
  01.05.95 1999 4500 Zinsen   BOT 010595 1M   80000
  \end{verbatim}

\end{itemize}

Solche Aufteilungen über ein Zwischenkonto oder die Kasse sind nötig, um
insbesondere auf den Bankkonten nur die Beträge auszuweisen, die auch im
Kontoauszug der Bank erscheinen werden ( = leichtere Kontrolle).

Schließlich wird empfohlen, immer die gleichen Formulierungen zu gebrauchen,
die so kurz wie möglich sein sollten (z.B. / = für, usw.). Bei wiederkehren-
den Vorgängen sollte man deshalb immer erst im Journal weiter oben nach-
schauen.

\chapter{Referenzen}

Eine Testplattform is auf 
\htmladdnormallink{http://fibu.adlp.org}{http://fibu.adlp.org} aufgebaut.

\end{appendix}

\end{document}
