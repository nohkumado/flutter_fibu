%*****************************************************************************
% copyright Winfried and Bruno Böttcher
\documentclass[12pt]{report}
\usepackage[french]{babel}
\usepackage[utf8]{inputenc}
\usepackage{epsf}
\usepackage{html}

\pagestyle{headings}
%*****************************************************************************
%*****************************************************************************
%\originalTeX
%\input{psfig}                     %solange der postscript includer da ist
%\germanTeX
%%%%%%%%%%%%%%%%

\def\figurepath{./}
\def\psfigurepath#1{\edef\figurepath{#1}}
%\def\docpath{scsi::4.$.word.texfiles.diplarbeit.bilder.}
%\def\docpath{/home/bboett/tex/gb.dipl/bilder/}

%%%%%% aus psfig

%\psfigurepath{scsi::4.$.word.texfiles.diplarbeit.bilder.}
%\psfigurepath{/home/bboett/tex/gb.dipl/bilder/}
%*****************************************************************************
% pageheight = 29.7cm (= DIN A4)

\topmargin-0.7cm                    % oberer Rand=3.7 +tm=2cm
\headheight1cm                     % Hoehe  der Kopfzeile 1cm
\headsep0.3cm                       % Abstand zur Text/Kopfzeile
\topskip1cm                         % Abstand Textanf/Text
%\footheight0.5cm                    % Hoehe Fu"sleiste .5cm
\textheight23cm                     % L"ange Seite
\footskip2cm                        % Tiefe Fu"szeile
% pagewidth =  21.0cm (= DIN A4)
\evensidemargin1cm                  % linker Rand = 3 +1 =4cm
\oddsidemargin1cm                   %    "                 "
\textwidth15cm                      % Textbreite





%*****************************************************************************
% Centred scaled captioned labeled figure
% example: \psfig{file.ps}{scale}{caption}{reference}
% \def\psfig#1#2#3#4{
%        \begin{figure}
%                \def\epsfsize##1##2{#2##1}
%                \centerline{\epsfbox{#1}}
%                \caption{#3}
%                \label{#4}
%        \end{figure}
% }

\usepackage{myepsf}
\def\psfig#1#2#3#4{
       \begin{figure}
               \setepsfsize{#2}\centerline{\epsfbox{#1}}
               \caption{#3} \label{#4}
       \end{figure}
}

%
%\newcommand{\postfig}[4]{       %for use with psfig
%   \begin{figure}[hbtp]
%   \centerline{\hbox{\psfig{figure=#1, height=#2}}}
%   \caption{#3}\label{#4}
%   \end{figure}}



%\input{\docpath hyph_ge}
%selb-st"an-dig
                                                                            
%*****************************************************************************
%\newtheorem{definition}{Definition}[chapter]
%*************** Damit die Bilder da bleiben wo ich sie hingepackt habe ******
\renewcommand{\topfraction}{0,9999}
\renewcommand{\bottomfraction}{0,9999}
\renewcommand{\textfraction}{0}

%*****************************************************************************

                                                                
%*****************************************************************************
\begin{document}

\tableofcontents

\chapter{Introduction}

\section{Pourquoi une comptabilité financière ?}


Pour l'administration des procédures financières une méthode de comptabilité
financière est essentielle. jFibu est un
logiciel qui fonctionne selon le principe de la double comptabilité.


La double comptabilité a été décrit pour la première fois en 1445 par Luca
Paciolo et est basée sur une idée très simple: ''Chaque operation est
représentée par un versement ou mouvement d'argent, qui procède à partir d'une
source vers une cible.'' Les sources et les cibles (appelés les comptes)
peuvent être considérés aussi comme des pots arrangés sur une grande table,
d'ou l'on retire de l'argent (des sources), ou bien dans lesquels de  l'argent
est déposé.  Arithmétiquement le retrait d'argent signifie une soustraction
(-), et le dépot une addition (+).  

\begin{verbatim}
      Cible                     Source
      +      <--- Transaction ---   -
\end{verbatim}


Chaque transaction concerne toujours deux comptes, c'est pour cela que le
procédé s'appelle également double comptabilité (partita doppia).


Le plan des comptes contient un ensemble de comptes librement choisis, ils sont
nécessaires pour la description des différentes transactions.  Le plan des
comptes peut être élargi à tout moment selon les exigences.  La destruction d'un
compte pendant une periode d'exercice est autorisée seulement si aucun mouvement
n'a été fait en utilisant ce compte.


Les comptes sont divisés en 4 groupes:



\begin{itemize}
\item Comptes actifs (argent comptant, comptes bancaires, valeurs immobilières,
capitaux, etc...)

\item Comptes passifs (fortune, posessions, dettes, etc.)

\item Dépenses

\item Revenus

\end{itemize}



Un certain sens de l'abstraction est nécessaire pour comprendre ou croire que
les sommes de ces groupes ont des signes changeants dans le contexte de la
comptabilité:

\begin{itemize}
\item Les comptes actifs sont positifs, 

\item Les comptes passifs sont négatifs,

\item Les dépenses sont positives,

\item Les revenus sont négatifs.

\end{itemize}

Ceci a comme conséquence que la somme globale de tous les comptes est zéro.

Le journal est un enregistrement séquentiel de toutes les transactions.  La
tâche importante du comptable consiste à découper la transaction en
sous-mouvements appropriés et à déterminer les comptes impliqués. En cas de
doute un nouveau compte peut être ouvert à tout moment.



\subsection{L'extrait de compte:} 

Tous les mouvements enregistrés dans le journal, concernant un certain compte,
peuvent être récapitulés sur une fiche de compte, et à partir du montant
initial et des mouvements effectués le montant total actuel du compte peut être
calculé.  Le degré d'expression de cette compilation dépend de la granularité
du plan des comptes (plus la granularité est fine, plus sélectives sont les
distinctions permises par le plan ).

\subsection{Bilan / compte de résultat:} 

A ne pas confondre.  Le bilan est la comparaison des soldes de tous les
 comptes actifs et de passif en un certain jour.  Il informe au sujet de la
 situation financière actuelle.

Le compte de résultat est la comparaison des soldes de tous les comptes de
dépense avec ceux des comptes de revenus, sur une certaine période.  Le calcul
du tableau informe au sujet de l'évolution économique.

\subsection{Valeurs initiales:} 

Au début d'un exercice comptable (nouveau journal) seulement les comptes
actifs et passifs obtiennent des valeurs initiales, de telle manière que la
somme des comptes actifs positifs soit égale à la somme des comptes passifs
négatifs.  



\subsection{Contrôlabilité:} 

Puisque chaque transaction concerne toujours deux
comptes, une fois + (cible), une fois - (source), la somme de tous les comptes
doit ètre zéro à tout moment.

La différence entre les comptes actifs et passifs est égale à la différence
entre les revenus et des dépenses, toutefois avec le signe opposé (c'est
nécessaire si la somme de tous les comptes doit ètre zéro).



\section{Pourquoi jFibu?}

En lisant des livres sur la comptabilité, on peut avoir l'impression que les
règles accompagnant la comptabilité, qui sont partiellement exigées par la loi,
rendent les choses plus compliqués, que le sujet l'exige.

Les nombreux logiciels de comptabilité actuellement sur le marché portent ce ballast avec eux, et sont donc en conséquence assez complexes. 

Afin d'implémenter facilement les relations enoncées ci-dessus dans un logiciel
simple, le développement d'un nouveau logiciel est devenu nécessaire (par
exemple il est possible de corriger les entrées dans le journal, les entrées
sont automatiquement triées selon la date).  Ceci assure une  comptabilité plus
claire, au détriment de certaines impositions par la loi concernant les grandes entreprises.

Des efforts ont été faits d'écrire le programme de telle manière qu'il puisse
être facilement porté sur différentes plateformes informatiques.



\subsection{Utilisateurs potentiels de  jFibu:} 

 Particuliers, comptables d'associations ou petites compagnies sans contraintes légales.


\chapter{Mise en place}

\section{utilisation de jFibu: }

\begin{itemize}

\item le programme  est lancé à partir d'une ligne de commande;

\item les nouvelles données pour le journal ou le plan des comptes sont saisies
par le clavier;  possibilité pour copier des données de la ligne précédente
(pas encore mise en application).

\item tandis qu'en comptabilité manuelle les  entrée du journal doivent ètre
répercutées sur les deux fiches de compte (calcul de +Saldo), ici
il suffit de lancer l'exécution (point de menu: exécution). L'on obtent ainsi un fichier avec toutes les informations nécessaires (plan des comptes,
journal, table des devises, bilan et  compte de résultat).

\item les résultats pourront ètre consultes à l'écran (Tabpanel pour les
composants, l'équilibre différents) de temps en temps ou une copie
imprimée de bookkepping est nécessaire et peut être fait (COPIE, pas encore
instrument de menupoint).

\item sur la sortie du programme l'état actuel est stocké automatiquement, mais
pour la sûreté une copie sur à disque souple devrait être tirée à la main;  on
devrait utiliser 2 disques souples alternatingly et toujours permuter le
dernier avec précéder pour garder les deux dernières situations.  Ainsi en cas
d'échec du principal-disque ou du travail entier de PC peut être continué à
tout moment sur un autre PC.

\item Les comptes en devises étrangères sont possibles, par contre les méthodes
de conversion automatiques ont été abandonnées, le problème étant un peu trop
complexe pour le moment, et nécessiterait de la part de l'utilisateur un suivi
continu;


\item Facilité d'entrée de données, en proposant des transactions standard
préremplies;
\end{itemize}

\begin{itemize}
\item Programmé dans JAVA pour toutes les plateformes d'informatique, qui soutiennent la machine virtuelle de Java (SUN-JVM):  Pc-dos, Archimedes, W D'Unix..;

\item nombre maximum des entrées seulement dépendantes de la capacité de la
base de données;


\item La langue et les menu-titres d'interaction sont librement (allemand, italien, Français...) des fibu*properties sélectionnables.



\item Rendre compte-plan librement sélectionnable;



\item Comptes de devise étrangère soutenus;



\item Rédacteur de ligne pour l'entrée de données contexte-référée;



\item Configuration des clefs d'édition au-dessus du dossier externe (fibu.cfg);  (pas encore mis en application)



\item Traitement de toutes les données dans la mémoire;



\item nombre maximum des entrées seulement dépendantes de la capacité de la mémoire;



\item Consultation des résultats sur l'écran (économie de papier!)  - ceux-ci restent et peuvent être employés encore au prochain traitement;



\item Imprimant sur LPT1 possible sur des imprimeurs avec 90 caractères par la ligne maximum (paysage normal de manuscrit DIN A4, ou portrait de 16cpi DIN A4);  Ajustement d'imprimeur au-dessus du dossier externe (fibu.prn);  (pas encore instrument)



\end{itemize}




\subsection{ Préalables à ce programme:
}



\begin{itemize}
\item N'importe quelle plateforme Java-avertie peut être employée, puisque la balancer-bibliothèque était au moins 32MB utilisé sont recommandées.



\item Disque non démontable suggéré - environ 700 K bytes sont nécessaires pour le programme;



\item Imprimeur utile (paysage 16cpi ou colonne 120 de papier);



\end{itemize}

\chapter{Manuel d'utilisation}



\subsection{ dossiers de jFibu:}



\begin{itemize}
\item classe de démarreur de xF pour la version du graphique jFibu.



\item messages et menus de fibu\_de\_DE.properties en allemand



\item messages et menus de fibu\_it\_IT.properties en italien



\item fibu.cfg éditant la table de devise



\item ajustements d'imprimeur de fibu.prn - les ordres d'évasion dans dossiers de la notation NOTA: de la HP PCL ces sont dans le même annuaire



\item  <plan des comptes de Name>.kpl pour la comptabilité < nom >, par exemple ELKI93.kpl



\item < journal de Name>.jrl pour la comptabilité < nom >, par exemple ELKI93.jrl



\item < Name>.lst des résultats pour la comptabilité < nom >, par exemple ELKI93.lst 



\end{itemize}

\begin{appendix}

 N.B.  ces dossiers tous sont localisés dans le même annuaire.

\chapter{Formats de fichiers}

\subsection{ Kpl de format de dossier}

La commande latérale de la position 01 (le nouveau côté de P, vident
séquentiellement) du chiffre vide du numéro du compte 02 03 1-4, par exemple
1101 07 08 titres de compte vides 50 indications, par exemple encaissent 58
indications de la devise 2, par exemple à gauche, DM, télex, etc.
(actuellement pas le verw.)  60 quantités pour l'évaluation (0 avec des comptes
d'active/passive), 12 caractères 72 librement 

 <PRE> La commande latérale de la position 01 (le nouveau côté de P, vident séquentiellement) du chiffre vide du numéro du compte 02 03 1-4, par exemple 1101 07 08 titres de compte vides 50 indications, par exemple encaissent 58 indications de la devise 2, par exemple à gauche, DM, télex, etc. (actuellement pas le verw.)  60 quantités pour l'évaluation (0 avec des comptes d'active/passive), 12 caractères 72 librement </PRE>


\subsection{ Jrl de format de dossier}

Position 01 indications de la date 8, par exemple numéro du compte 01.01.1993
09 10 vide +, (inférieur), 1-4 numéro du compte 15 vide des caractères 14 -,
(source), 1-4 le procédé 20 vide des caractères 19 des caractères de maximum 45
65 indications de la devise 2 (comme avec KPL) 67 s'élèvent jusqu'à 12
caractères 79 librement;  le programme accroche la dernière position dans le
format de rendement dessus dans cette position l'équilibre = 12 caractères 90

 <PRE> Position 01 indications de la date 8, par exemple numéro du compte 01.01.1993 09 10 vide +, (inférieur), 1-4 numéro du compte 15 vide des caractères 14 -, (source), 1-4 le procédé 20 vide des caractères 19 des caractères de maximum 45 65 indications de la devise 2 (comme avec KPL) 67 s'élèvent jusqu'à 12 caractères 79 librement;  le programme accroche la dernière position dans le format de rendement dessus dans cette position l'équilibre = 12 caractères 90 </PRE>

 Dans la compatibilité de jFibu / * le EOF est ignoré. 


\chapter{Concepts de comptabilité}
\subsection{ Valeurs initiales}

 À 01.01.9X les valeurs initiales doivent être indiquées aux comptes actifs et de passif (prétendus transferts).  Ceci est réalisé en insérant des journal-lignes où seulement un compte est indiqué, le compte positif, ou le compte passif négatif, alors que le deuxième compte est établi exeptionally à 0.

 Exemple:

 {\tt  01.01.93 1101 0 argents comptants de transfert 100000 01.01.93 0 2100 transfèrent Patrimonio 100000 }



\subsection{ Références pour enregistrer conserver
}

 Pour la démonstration le plan raccourci suivant de compte est employé:

{\tt  
1001 compte de compte courant de l'argent comptant 1010 1020 dépôts à terme:  Le crédit au camarade X A OFFERT à 1100 1999 comptes de suspence 0 fins d'Aktivas 2001 capacités d'abilityability 2010 dettes de l'arrière 2100 de réservation à la fin du fournisseur 0 des responsabilités 3001 administratives que les dépenses 3010 coût énergetique 3020 le service coûte 0 fins de coûte 4100 revenus:  des membres 4500 revenus:  Revenus de l'intérêt 4600:  diverse 0 fins des revenus }

 Beaucoup de procédures consistent en réalité de plusieurs pas à pas, qui doivent être enregistrés séparément dans le journal:



\begin{itemize}
\item différents revenus (cxxx = dossier de nombre, dossier etc...)  {\tt  01.02.95 1001 4100 de x, cxxx 100000 15.02.95 1001 4100 de y, chèque, cxxx 500000 01.03.95 1010 1001 de l'argent comptant aux chèques postaux 600000 }



\item calcul détaillé, paiement plus tard (dxxx = nombre du dossier, de dossier etc...)
{\tt  01.02.95 3010 2100 calcul ENEL 12-1.95, dxxx 300000 15.03.95 2100 1010 à ENEL Ueb/Scheckxxx 300000 }



\item calcul détaillé, paiement immédiatement, affectation des coûts  {\tt 
01.02.95 1999 1010 Rech.Telecom 1bim95, dxxx 600000 01.02.95 3020 1999 services 450000 de Rech.Telecom 1bim95 01.02.95 1100 1999 Rech.Telecom 1bim95 en privé 150000 15.02.95 1001 1100 de x / de priv.Telefon 150000 }



\item Abgezinst de Festgeldanlage  {\tt  010595 1M 920000 ONT OFFERT l'usine 01.02.95 1020 1010

01.05.95 1010 1999 Rueckz.  010595 1m Ont offert 1000000 01.05.95 1999 1020 Rueckz.  010595 1M ONT OFFERT 920000 01.05.95 1999 4500 que l'intérêt A OFFERT 010595 1M 80000 }



\end{itemize}


 De telles attributions au-dessus d'un suspence rendent compte ou l'argent
 comptant sont nécessaire, dans l'ordre en particulier pour prouver sur les
 comptes bancaires seulement les montants, qui apparaîtront à la banque
 également dans le rapport de compte (= une commande plus facile).

 Enfin recommandé d'employer toujours les mêmes formulations soyez que possible
 pendant que sous peu si (par exemple / = pour, etc.).  Avec des procédures se
 reproduisantes on devrait vérifier donc toujours seulement dans le journal
 plus loin en haut.

\end{appendix}
\end{document}
