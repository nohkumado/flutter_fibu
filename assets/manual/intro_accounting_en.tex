\chapter{Accounting concepts}

\section{ Initial values}

 At the 01.01.9X initial values must be given to the active and passive
accounts (so-called transfers).  This is achieved by inserting journal-lines where only one account is designated, either the positive account, or
the negative passive account, while the second account is set exeptionally to 0.

 Example:

 {\tt  01,01,93 1101 0 transfer cash 100000 01,01,93 0 2100 transfer
Patrimonio 100000 }



\section{ References to record keeping}

 For the demonstration purposes the following shortened account plan is used:

{\tt  
1001 cash 
1010 current account account 
1020 time deposits:  Credit at
comrade x OFFERED 
1100 
1999 suspence account 
0 end of Aktivas 
2001
abilityability abilities 
2010 reserve rear 2100 debts at supplier 0 end of
liabilities 3001 administratives expense 3010 energy costs 3020 service costs 0
end of the costs 4100 incomes:  of members 4500 incomes:  Interest 4600
incomes:  various 0 end of the incomes }

 Many procedures consist in reality of several single steps, which must be
registered separately in the journal:



\begin{itemize}

\item different incomes (cxxx = number file, file etc.)  {\tt  01,02,95 1001
4100 of x, cxxx 100000 15,02,95 1001 4100 of y, cheque, cxxx 500000 01,03,95
1010 1001 from cash to giro 600000 }



\item detailed calculation, payment later (dxxx = number of the file, file etc.)
{\tt  01,02,95 3010 2100 calculation ENEL 12-1,95, dxxx 300000 15,03,95 2100
1010 at ENEL Ueb/Scheckxxx 300000 }



\item detailed calculation, payment immediately, allocation of the costs  {\tt 
01,02,95 1999 1010 Rech.Telecom 1bim95, dxxx 600000 01,02,95 3020 1999
Rech.Telecom 1bim95 service 450000 01,02,95 1100 1999 Rech.Telecom 1bim95
privately 150000 15,02,95 1001 1100 of x / priv.Telefon 150000 }



\item Festgeldanlage abgezinst  {\tt  010595 1M 920000 OFFERED 01,02,95 1020 1010
plant

01,05,95 1010 1999 Rueckz.  010595 1M OFFERED 1000000 01,05,95 1999 1020
Rueckz.  010595 1M OFFERED 920000 01,05,95 1999 4500 interest OFFERED 010595 1M
80000 }



\end{itemize}

 Such allocations over a suspence account or the cash are necessary, in
order in particular to prove on the bank accounts only the amounts, which will
appear to the bank also in the account statement (= easier control).

 Finally recommended to always use the same formulations as possible be as
short should (e.g. / = for, etc.).  With recurring procedures one should check
therefore always only in the journal further above.

