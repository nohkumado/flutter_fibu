\chapter{Installation}

\section{Caracteristics of bb-Fibu:}

\begin{itemize}

\item Programmed in standard PHP4/SQL, compatible with all operating systems where the apache and postgresSQL servers run, or more generally where GNU software runs, virtually any somehow recent operating system.

\item selection of an operative language through a ressource file.

\item free definable accounting plan

\item foreign currency accounts possible, the automatic conversion parts are disabled though, since the handling of this is quite a tough thema.

\item ease of acquisition of data through the proposal of prefilled template entries.

\item maximum number of journal entries limited only by the capacities of the database.

\item on screen consultation of all parts of the accounting, caching of
previous result.

\end{itemize}

\subsection{ Prerequisites for this program:}

To run this program several assertions have to be fulfilled:
you need:

\begin{itemize}


\item a recent \htmladdnormallink{apache}{http://www.apache.org/} daemon
\item a recent \htmladdnormallink{PHP4}{http://www.php.net/} module for the apache
\item a \htmladdnormallink{PostgresSQL}{http://www.postgresql.org/} database
\item the php2postgres module/library
\item the perl procedural language for postgres
\item the account where the fibu will be installed needs .htaccess control and at least the 'AllowOverride Options' right.


\end{itemize}

\subsection{Installation}

After unpacking the fibu tarball: 

\begin{itemize}

\item  check if pgsql is installed fixed or as a loadable module on your apache
install, eventually add/decomment a dl(pgsql.so); on top of the
includes/global.php, after '\$globalloaded = ''true'';'

\item  create a dabatase for the fibu system
  e.g. <bold>createdb fibu</bold>, assert that plperl is loaded (eventually add it with createlang plperl)

\item  cp htaccess-sample to .htaccess, adn the other includes/*-sample.php to includes/*.php, edit them and adjust them to your needs

\item  edit includes/settings.php accordingly to your needs (especially the
  defaultadmin line, that's the one that gives the user with that name the
  admin rights to access the system.

\item  check that in the .htaccess file the FilesMatch sections are all commented out

\item  load the file admin/install.php through the server into a browser.
  <bold>lynx http://localhost/<PATHTOFIBU>/admin/install.php</bold>
  This will populate the DB with the necessary tables and fill the needed tables.

\item  edit your .htaccess file and decomment the FilesMatch section to protect access to the only index.php file

\item  log into your new fibu system with the first mentioned admin name and the
  passwd 'tochange', first of all, change the admin passwd in the options
  section. 

\item  create a new book set and

\item  set up your account-plan, set up some transaction templates, and you are off
with your brand new book-keeping system. don't forget to make regular backups
of your database, e.g. through a cron-script (<bold>crontab -e -> 0 0 * * * cd
                /PATHTOFIBU/;cp DBBAK DBBAK.old;pg\_dump fibu >DBBAK</bold>)
\end{itemize}

