\chapter{Introduction}

\section{ Why financial accounting?}

For the administration of financial procedures a financial accounting method is essential.  Stock-accounting bb-Fibu is an EDP program, which works according to the principle of the double account processing.

The double account processing was described for the first time 1445 by Luca
Paciolo and been based on a conceivablly simple idea: ''Each procedure is
represented as a cash transaction, which proceeds from a source to a
target.'' Sources and targets (called accounts) can be thought of also as
pots, which all stand on a large table, and out of whose money is
taken (sources), or into whose the money is inserted.  Arithmetically seen
the withdrawal of money means a subtraction (-), and inserting an addition (+).

  
 
\begin{verbatim}
      Lower                    source 
      +     <--- procedure --     -
\end{verbatim}


Since each procedure concerns always 2 accounts, the procedure is called also double account processing (partita doppia).


The account-plan contains a freely selectable quantity of accounts, as they are
needed for the description of the different transactions.  The account-plan can
be extended at any time as required.  Taking of an account away in the midst of
one financial year is permitted only if no movements were made using this
account.


The accounts are divided into 4 groups:



\begin{itemize}\item Assets accounts (cash, bank accounts, real estate values, assets, etc.)



\item Passive Accounts(fortune, posessions, debts, etc.)



\item Expenditures



\item Incomes



\end{itemize}


Some abstract thinking is necessary to understood or believe that the balances
of these groups have changing signs in the context of the bookkeeping:



\begin{itemize}

\item Assets accounts are positive,

\item Passive accounts are negative,

\item Expenditures are positive,

\item Incomes are negative.

\end{itemize}


This results in a global balance sum of zero as control.


The journal is a sequential recording of all procedures.  The important task of
the accountant consists of dividing the procedure evt. into its single steps
and determining the associated accounts.  In the case of doubt a new account
can be opened at any time.



\subsection{The account statement:}

All movements, which concern a certain account, registered in the journal, can
be summarized on a ledger card, and from the initial amount and the movements
the current total amount of the account can be calculated (balance).  The force
of expression of this compilation depends only and alone on the grain-size of
 the account plan (fine tuning the graininess of the plan permits more selective distinctions).



\subsection{ Balance / account:} 

 May not be confounded.  The balance is the comparison of the balances of
all active and passive accounts on a certain day.  It informs about the present
financial situation.

 The account is the comparison of the balances of all expenditure accounts
with those of the income accounts, over a certain period.  The restaurant
calculation informs about the economic behavior.



\subsection{ Initial values:} 

At the beginning of the accounting period (new journal) only the active and
passive accounts get initial values, in such a way that the sum of the positive
assets accounts is equal to the sum of the negative passive accounts.

\subsection{ Controllability:} 

 Since the sequentially added movements concern always 2 accounts,
once + (target), once - (source), the total sum of all accounts equals zero all
the time.

 The difference between active and passive accounts is in the amount equal
to the difference between incomes and expenditures, however with opposite sign
(with is necessary if  the sum total of all accounts has to be zero).

